\chapter{September - 2012} % Chapter title
\label{ch:sep:2012} % For referencing the chapter elsewhere, use \autoref{ch:introduction} 

%----------------------------------------------------------------------------------------

\section{SublimeText 2, Crack}
Although I want to pay the software 'SublimeText', the price is too expensive for me.
I have to search the crack over the Internet, the crack solution is:
\begin{enumerate}
\item  Use the vim to open the programe file as binary format;
\item  Find the string "3542 4535 3039 4333 3342 3032 3031 3131";
\item  Change '3342' to '3242';
\item  Then, using the following license to register.
\end{enumerate}
\begin{comment}
\end{comment}
\begin{verbatim}
-----BEGIN LICENSE-----
Zhiwei
Unlimited User License
EA7E-27304
295F06AD49A222300142F52385B79017
0EBC472E766824A3E764A77BEE8905AB
532EA109E190C3E75483FF14D0010DFA
5F532494C59E830CCE5002A0BE9110EC
0B2AC29AB5BEAAA3242124FD2B99A2CE
F443912E3A653CDE9195B7C15D5ED807
F7C3904D5473C89529796E723C79C019
3809B28DBBF4E34CAC38938A8389B925
-----END LICENSE-----
\end{verbatim}
\hfill {\tiny edited on 2012-09-23}

\section{Vim, Edit, Binary File}
Vim can edit a binary file easier than the editor "Emacs". 
You need the tool 'xxd' to dump a file from binary format to text format.
After editing, you use 'xxd' to revert the file format into binary. 
For example,

\begin{verbatim}
vi -b bin_file_name
:%!xxd
....
editing something
....
:%!xxd -r
:wq
\end{verbatim}
\hfill {\tiny edited on 2012-09-23}

\section{Vim Crash, Input Method, Chinese Character Directory Name}
The VIM, SublimeText2 and Notepad++ crashed after I installed a new WB98 word bank for JiDianWB.
I wondered why and how the installation made this happen. 
Finally, I found the reason.
The reason is that the directory name of the new word bank is spelled as Chinese characters.
Since my computer OS is the Windows 7 English version, the input method manager seems not to handle the Chinese Directory Name well.
The solution is very simple, changing the directory name from Chinese characters to English letters.
\hfill {\tiny  edited on 2012-09-17}

\section{Python, Urlopen}
Today, the Website of dictionary, "iCiba", blocked my access through Python scripts after my looking up several words. 
After several trials, I found the solution: If using the standard usage for HTTP protocols, the problem can be solved.
The codes in the following:

\par From 
\begin{verbatim}
    try:
        page = urllib2.urlopen(g_collins_url+word)
    except urllib2.HTTPError, e:
        print g_collins_url, e.code
        return False
    soup = BeautifulSoup(page)

\end{verbatim}

\par To
\begin{verbatim}
    try:
        req = urllib2.Request(g_collins_url+word)
        response = urllib2.urlopen(req)
    except urllib2.HTTPError, e:
        print g_collins_url, e.code
        return False

    page = response.read()
    soup = BeautifulSoup(page)
\end{verbatim}
\hfill {\tiny  edited on 2012-09-19.}
%
\section{Combine Picture for Longman Words}
Since the LongmanCN dictionary splits one word picture into several frames,
I have to make a small program script to combine them, like this:
\begin{verbatim}
  if match:
        fp = open(str(word_dir+file),'rb')
#       print file;
        old_image = Image.open(fp)
        y = old_image.size[1];
        new_image.paste(old_image, (x_index, y_index));
        fp.close();

        x_index = 0;
        y_index = y_index + y;
        os.remove(word_dir+file)
\end{verbatim}
It is noted that the picture file should be closed if you remove or delete it. 
Since the IMage library does not support the method 'close',
you have to take an alternative way to handle this.
\hfill {\tiny  edited on 2012-09-19.}


%%%%%%%%%%%%%%%%%%%%%%%%%%%%%%%%%%%%%%%%%%%%%%%%%%%%%%%%%
