\chapter{May - 2012} % Chapter title

\label{ch:may:2012} % For referencing the chapter elsewhere, use \autoref{ch:introduction} 

%----------------------------------------------------------------------------------------
\section{Reading Style Lessons}
\marginpar{Have something to say, and say it as clearly as you can. That is the only secret of style. \\
-Mathew Arnold}
\subsection{Clarity of an Article}
Someone criticizes you that your proses are not understandable. You  bewilder your readers because you can not organize your ideas coherently.  
People presume that the complex sentences with abstract and long words are equal to deep thinking. In fact, such sentences mean that you do not understand well what you write. So you have to hide the evidence from readers and even yourself.  
As a matter of fact, writing and thinking can help with each other.

\subsection{Correctness}
The rules are specific and the principles are abstract.
\subsubsection{Ruler}
There are three kinds of rulers, real rules, social rules and invented rules. The invented rules includes Folklore and Elegant.

The Folklore options are shown in the following:
\begin{enumerate}
\item The sentences began with \texttt{'And'} and \texttt{'But'} are acceptable.
\item \texttt{'Because'} and \texttt{'Since'}. The positions of two words are dependent on the prior knowledge of readers. The order of sentences are so important that we focus on it. We should put the familiar material in the first place and the strange ones later.
\item \texttt{'that'} and \texttt{'which'} in restrictive or non-restrictive clauses. Which is your choice depends on the emphasis of your idea. 
\item \texttt{'fewer'} and \texttt{'less'}. No one uses \texttt{fewer} with mass nouns (fewer dirt) but educated writers often use less with countable plural nouns (less resources).
\item \texttt{'since'} and \texttt{'while'}. The two words mean \texttt{because} and \texttt{although} if the clauses are more familiar to readers.  
\end{enumerate}
The Elegant options are:
\begin{enumerate}
\item {\bf Don't split infinitives ?} 
\begin{itemize}
\item[] $\surd$ to slightly conceal
\item[] $\surd$ to conceal slightly
\end{itemize}
\item{\bf whom or who?}
\begin{itemize}
\item []$\surd$ Who am I writing for?
\item []$\surd$ For whom am I writing?
\end{itemize}
If the relative clause modifies a noun and you can delete the relative pronoun and still make sense, the correct form is whom:
\begin{itemize}
\item []$\surd$ The committee chose someone whom they trusted.
\item []$\surd$ The committee chose someone [ ] they trusted.
\end{itemize}
If you cannot delete the \texttt{who/whom}, the correct form is \texttt{who}:
\begin{itemize}
\item []$\surd$ The committee chose someone {\bf who} earned their trust. 
\item []$\times$ The committee chose someone [ ] earned their trust.
\end{itemize}
Two exceptions:(1)you cannot delete \texttt{whom} when it begins a clause that is the object of a verb. In that case, you have to depend on the grammar of the clause:
\begin{itemize}
\item[]$\surd$ The committee decided \texttt{whom} they should choose.
\item[]$\surd$ The committee decided \texttt{who} was to be chosen.
\end{itemize}
(2)Always use whom when it is the object of a preposition:
\begin{itemize}
\item[]$\surd$ The committee choose someone in \texttt{whom} they had confidence. 
\end{itemize}

\end{enumerate}
\hfill {\tiny P.26, edited on 2012-05-01.}

\section{Crop PDF, Foxit Phantom}
What you do is: open the file that you want to crop, select menu, <Organize>-><Crop Pages...>, and setup the inch of margins and pages index. All things are down.
\hfill {\tiny  edited on 2012-05-02.}

\section{Reading Lessons of Clarity}
P.27\\
\begin{itemize}
\item Don't end a sentence with a preposition.
\item Use the singular with \texttt{'none'} and \texttt{'any'}.
\end{itemize}
Hobgoblins:
\begin{itemize}
\item Never use \texttt{'like'} for \texttt{'as'} or \texttt{'as if'}.\\
For example: \\
$\surd$ These operations failed like the earlier ones did.\\
$\surd$ These operations failed as the earlier ones did.
\item Don't use \texttt{'hopefully'} to mean \texttt{'I hope'}.\\
For example:\\
$\surd$ Hopefully, it will not rain.\\
$\surd$ I hope that it will not rain.
\item Don't use \texttt{'finalize'} to mean \texttt{'finish'} or \texttt{'complete'}.
\item Don't use \texttt{'impact'} as a verb, but as a noun.
\item Don't modify absolute words such as \texttt{ perfect, unique, final complete} with \texttt{ very, more
    quite}, and so on.
\end{itemize}

Some Words That attract Special Attention\\
aggravate, anticipate, anxious, cohort, comprise, continuous, disinterested, enormity, fortuitous, fulsome, 
notorious.

P.38\\
Two principles:
\begin{itemize}
\item Its main characters are subjects of verbs.
\item Those verbs express specific actions.
\end{itemize}
P.42\\
Readers will think your writing is dense if you use lots of abstract nouns, especially those derived from verbs and adjectives,
nouns ending in -tion, -ment, -ence, and so on, especially when you make those abstract nouns the subjects of verbs.

A noun derived from a verb or adjective has a technical name: nominalization. The word illustrates its meaning: When we nominalize nominalize, we create the nomialization nominalization.

\section{Multiple clipboard, Vim}
The multiple clipboard is provided by Vim. The function can make us work more efficiently.  The buffer of clipboard has its name, which looks like ("b). The symbol means a buffer named \texttt{b}. If you want to save the current line into the buffer b, press {\bf "byy}. After that, you paste the content in buffer b into wherever you assign. The instruction is: {\bf"bp}. 

\section{Clarity in English Writing}
\marginpar{You write your prose for your readers.}
Choosing between Active and Passive P.65
\begin{enumerate}
\item Must your readers know who is responsible for the action?
\item Would the active or passive verb help your readers move more smoothly from one sentence to the next?
\item Would the active or passive give readers a more consistent and appropriate point of view?
Some writers switch from one character to another for no apparent reason. You should avoid this.
\end{enumerate}

When academic writers do use the first person, however, they use it in certain ways. There are two kinds:
\begin{itemize}
\item One kind refers to research activities: \texttt{study, investigate, examine, observe, use}. Those verbs are usually in the passive voice: 
\emph{ The subjects were observed $\ldots$} .
\item The other kind of verb refers not to the subject matter or the research, but to the writer's own writing and thinking: \emph{cite, show, inquire.}
These verbs are often active and in the first person: \emph{We will show $\ldots$ }. 
\end{itemize}
\hfill {\tiny  edited on 2012-05-02.}
\section{IPE draw tool, Image Conversion}
IPE is an excellent draw tool, which can create high quality vector figures. For example, it can generate the file with extension .eps or .pdf. These figures must be converted into bitmap format (.png) if you want to insert them into Microsoft Word. The conversion command line is:
\begin{verbatim}
Usage: iperender [ -svg | -png ] [ -page <page> ]
    [ -view <view> ] [ -resolution <dpi> ] infile outfile
Iperender saves a single page of the Ipe document in some formats.
 -page       : page to save (default 1).
 -view       : view to save (default 1).
 -resolution : resolution for png format (default 72.0 ppi).
\end{verbatim}

Or you want to export figures with PDF format.
\begin{verbatim}
Usage: ipetoipe ( -xml | -eps | -pdf ) <options> infile [ outfile ]
Ipetoipe converts between the different Ipe file formats.
 -export      : output contains no Ipe markup.
 -pages <n-m> : export only these pages (implies -export).
 -view <p-v>  : export only this view (implies -export).
 -lastview    : export only last view of each page (implies -export).
 -runlatex    : run Latex even for XML output.
 -nocolor     : avoid any color commands in EPS output.
 -nozip:      : do not compress PDF streams.
 -styleupdate : update style sheets.
\end{verbatim}
\hfill {\tiny  edited on 2012-05-04.}
\section{Latex, Table, Caption}
Sometimes, you need place the caption of a table below its content. You do it by the commands.
\begin{verbatim}
\begin{table}
\begin{tabular}
\end{tabular}
\caption{Example}
\end{table}
\end{verbatim}
\begin{table}[h]
\centering
\begin{tabular}{ccc}
\toprule
1 & 2 &3\\
\midrule
1 & 2 &3\\
1 & 2 &3\\
1 & 2 &3\\
\bottomrule
\end{tabular}
\caption{Example}
\end{table}
\hfill {\tiny  edited on 2012-05-04.}

\section{IEEE, Latex template, figure}
The IEEEtran document class will only center figure captions in Conference Mode, i.e. document class option "conference" is set.
(See also IEEEtrans documentation "How to Use the IEEEtran LATEX Class", page 2: "Conference Mode Details: Conference mode makes a number of significant changes to the way IEEEtran behaves.
\hfill{\tiny edited on 2012-05-04.}

\section{Latex, Fourier Macro package}
The package makes the fonts more graceful so that it is worth trying.
\begin{verbatim}
\usepackage{fourier}
\usepackage[OT1]{fontenc} %if there are CJK characters
\end{verbatim}
\hfill{\tiny edited on 2012-05-04.}
\section{Clarity in English Writing}
\begin{enumerate}
\item The terms active and passive, are ambiguous, because they can refer not only to those two grammatical constructions,
but to how a sentence make you feel.
\item Most readers want the subjects of verbs to name the main characters in a story 
and those main characters to be flesh and blood.
\item P.62 If you are in favour of something, you support it and think that it is good thing.
\item Reconstructing absent characters and remove nominlaization or abstract words.
\item P.79 We can achieve clarity just by mapping characters and actions onto subjects and verbs.
\item P.80 The last four words introduce an important character who is the subject of next sentence.
\item P.82 Begin sentences with information familiar to your readers. It includes a few words in the last sentence and what readers always know.
\item End sentences with information the readers can not anticipate. 
\item this, these, that, those, another, such, second, more
\item What is the difference between cohesive and coherence? The characters must belong to only one thing, one topic.
\end{enumerate}
\hfill {\tiny  edited on 2012-05-04.}
\section{Vim, words count}
\begin{center}
\bf press <g>, and then <ctrl>+<g>
\end{center}
\hfill {\tiny  edited on 2012-05-04.}
\section{EBOOK, Reading}
I download some books for further reading. They are:
\begin{enumerate}
\item \emph{English Grammar workbook for dummies}
\item \emph{Beyond feelings: A guide to critical thinking}
\item \emph{Using social Theory: Thinking through Research}
\end{enumerate}
\hfill {\tiny  edited on 2012-05-05.}
%
\section{Collins Dictionary, icon meaning}
\begin{itemize}
\item D, Dictionary
\item T, Thesaurus, same meaning words
\item U, Usage
\item G, Grammar
\item W, Word bank
\end{itemize}
\hfill {\tiny  edited on 2012-05-08.}
%
\section{Pip, Python Package Management}
\begin{enumerate}
\item Install a package, \begin{center} \$pip install <package-name> \\\$pip install <package-name>==<version>\end{center}
\item Update a package, \begin{center} \$pip install -upgrade <package-name> >= <version> (if you do not provide the version number, the latest version will be obtained.)\end{center}
\item Remove a package, \begin{center} \$pip uninstall <package-name> \end{center}
\end{enumerate}
\hfill {\tiny  edited on 2012-05-08.}
%
\section{Word 2007, Aurora, Equation, Update}
If you want to update all equations, select the menu <aurora, properties> from the ribbon of word menu and change some properties. The system will update the all equations in the document. And there are some bugs when you use the equation with index numbers, which cause the word system collapse.
\hfill {\tiny  edited on 2012-05-08.}
%
\section{Wrap lines, Vim}
We can set the limit of a line with the command such as 'set tw=$500$'. If you do not set a limit, just 'set tw=$0$'.
\hfill {\tiny  edited on 2012-05-09.}
%
\section{Xelatex, Fontspec Package}
Xelatex can use the package fontspec to assign a specific font as main font. (Tips: update l3kernel fontspec packages).
\begin{verbatim}
\documentclass[11pt,a4paper]{article}
\usepackage{fontspec}
\setmainfont{STSong}
% Chinese Line Wrapper
\XeTeXlinebreaklocale "zh"
\XeTeXlinebreakskip = 0pt plus 1pt
% Skip Two Chinese Characters
\makeatletter
\let\@afterindentfalse\@afterindenttrue
\@afterindenttrue
\makeatother
\setlength{\parindent}{2em}

\begin{document}
Chinese Characters
{\fontspec{STSong} Chinese Characters}
\end{document}
\end{verbatim}
\hfill {\tiny  edited on 2012-05-11.}
%
\section{Caption, Label, Latex}
Today, I confront a problem. The cross referencing number of all tables are wrongly linked to the section number in my article . After googling the web, the solution is found :
Put caption clause before label clause, such like:
\begin{verbatim}
\caption{Simulations Table} \label{tb:sim_res}
\end{verbatim}
\hfill {\tiny  edited on 2012-05-11.}
%
\section{Noun, Verb, Modifier, VIM}
Vim system has its unique grammar system. In short, it includes Noun, Verb, and Modifier. The position commands, like 'j', '0', '\$' and so on, are nouns. Verb are 'd'(delete), 'c'(change), 'y'(yank,copy) and 'v'(visual). Modifiers are 'i'(inside), 'a'(around), 't'(untill) and 'f'(find). 
\begin{center}
\bf Noun $+$ Verb $+$ Modifier $+$ Objects
\end{center}
Pay more attention to the modifier 'i' and 'a'.
\hfill {\tiny  edited on 2012-05-13.}
%
\section{log system, glog, google log}
Google log system is a excelent log system. The basic step is:
\begin{enumerate}
\item install, \begin{verbatim} ./configure; make; make install \end{verbatim}
\item add a function for sending email
\begin{itemize}
\item Install SendEmail; 
\begin{verbatim}apt-get install sendmail; 
apt-get install libio-socket-ssl-perl;
sendEmail -f username@foo.com -t destination@foo.com 
    -cc carboncopy@foo.com 
    -u "Message title" 
    -m "The body of the message" 
    -s smtp.foo.com:25 
    -xu username 
    -xp password \end{verbatim} 
\item Modify the logging.cc:1504
\begin{verbatim}string cmd = FLAGS_logmailer 
    + " -f sender.drumtm@gmail.com 
    -s smtp.gmail.com:25 
    -o tls=auto 
    -xu sender.drumtm 
    -xp 5hRZeEQGzn1asmL" 
    + " -u " + "\"" + subject + "\"" 
    + " -m " +"\"" + body +"\"" 
    + " -t " + dest; 
    \end{verbatim} 
\item Modify the logging.cc: 135 
\begin{verbatim}
GLOG_DEFINE_string(logmailer, "/usr/bin/sendEmail", 
    "Mailer used to send logging email"); 
\end{verbatim}
\end{itemize} 
\item re-install
\begin{verbatim} ./configure; make; make install \end{verbatim}
\item modify the cpp file
\begin{verbatim}
#include <glog/logging.h>
main()
{
    google::SetLogDestination(google::INFO,("./log.txt"));
    google::InitGoogleLogging("");
    google::SendEmail("to@gmail.com", "subject", "body");
    LOG(INFO) << "Found ";
    google::ShutdownGoogleLogging();
}\end{verbatim}
\end{enumerate}
\hfill {\tiny  edited on 2012-05-13.}
%
\section{ncurses, TTY, GDB}
\begin{enumerate}
\item Prepare two xterm windows, use the command {\bf tty} to identify the name of tty.
\item Redirect the screen information (such as printf)to tty1.
\begin{verbatim}
$GDB> tty /dev/tty1
\end{verbatim}
\end{enumerate}
\hfill {\tiny  edited on 2012-05-13.}
%
\section{Curses, Program}
Today, I study the program using a package called n-curses. 
Because I hope I can do a skeleton program for future codings. 
The skeleton program will include three parts: recording the log on local disk, sending email to developers and displaying a concise user interface. 
First two of them will be implemented by Google logging system. 
\par The final one will be done by the n-curses.
I did several trials and understood the mechanism of n-curses. 
The window in curses systems, in fact, is a block of memory, saving the responding data structures.
What we do is to display it on the terminal screen at the proper time.
The display routines are called {\bf wrefresh(WINDOWS*) }, which puts the characters that changes just now if the specific window are not active.
Thus, if you want totally clean refresh operations, first of all is to activate the windows by {\bf touchwin(WINDOW*)}.
Some examples are shown in the following:
\begin{verbatim}
switch (n) {
   case 1:
       touchwin(fix_win);
       wrefresh(fix_win);
       break;
   case 2:
       touchwin(scroll_win);
       wrefresh(scroll_win);
       break;
   case 0:
       touchwin(stdscr);
       wrefresh(stdscr);
       break;
   default:
       touchwin(stdscr);
       wrefresh(stdscr);
       break;
}
napms(500);
\end{verbatim}
\hfill {\tiny  edited on 2012-05-14.}
%
\section{Align, Program, C}
Left align:
\begin{verbatim}
printf("%-40s", str);
\end{verbatim}
Right align:
\begin{verbatim}
printf("%40s", str);
\end{verbatim}
\hfill {\tiny  edited on 2012-05-14.}
%
\section{Data Exchange, Json}
JSON (Java Script Object Notation) is a text format for data exchange. 
It is simpler than XML format.
JSON includes Object, Array, String, and Number Types.
Perhaps, I could use the format for future coding.
\hfill {\tiny  edited on 2012-05-15.}
%
\section{Separate, UI and Code}
In the last two days, I consider some problems for future coding. 
The problems consist of Logging and front-end user interface (UI) design. 
This morning, I am suddenly aware of the importance of loose connections between front-end UI and back-end programs. 
The back-end programs should be independent of the UI. 
They ought be connected only by a data structure like JSON.
\hfill {\tiny  edited on 2012-05-15.}
%
\section{UNIX, Command, Find}
Unix program principle: One Tool One work. 
Thus, you will combine the two commands with together.
\begin{verbatim}
find   /opt   -name '.svn'|xargs rm -r
\end{verbatim}
\hfill {\tiny  edited on 2012-05-17.}
%
\section{Ubuntu, Install, GCC}
\begin{verbatim}
$ sudo apt-get install build-essential
$ sudo apt-get install linux-headers-$(uname -r)
\end{verbatim}
\hfill {\tiny  edited on 2012-05-18.}
%
\section{Python, Html, QT}
I have been searching a Python library that handles html files with CSS. Today, I find my solution, QtWebKit, from QT library family.
\begin{verbatim}
import os, sys

import winsound
import time
# Import PySide classes
from PySide.QtCore import *
from PySide.QtGui import *
from PySide.QtWebKit import *

# Go through the dir "html"
class DisplayHtml(QWebView):
    def __init__(self, html_file_list):
        QWebView.__init__(self)
        self.counter = 1
        self.html_file_list = html_file_list
        self.file_num = len(self.html_file_list)       
        #self.load(QUrl("c:/tmp/my_codes/get_english_words/html/2012_03_07/woo.html") )
        self.show()
        self.timer = QTimer(self)
        self.connect(self.timer, SIGNAL("timeout()"), self.LoadHtml)
        self.timer.start(2500)
        self.stop_flag = 0
    def keyPressEvent(self, e):
        if e.key() == Qt.Key_Escape:
            self.close()
        if e.key() == Qt.Key_Space:
            self.stop_flag = self.stop_flag^1 
            if self.stop_flag == 1:
                self.timer.stop()
            else:
                self.timer.start()

    def LoadHtml(self):
        if self.counter < self.file_num:
                html_file = self.html_file_list[self.counter]
                html_file = html_file.replace('\\', r'/')

                self.setWindowTitle(html_file)
                self.setUrl(html_file)

                wav_file = html_file.replace('html','wav')
                ret = os.access(wav_file, os.F_OK)
                if True==ret:
                    wav = winsound.PlaySound(wav_file, winsound.SND_FILENAME)
                self.counter = self.counter + 1
        else:
            self.counter = 0 
            self.html_file_list.reverse()


def listdir_fullpath(d):
    return [os.path.join(d, f) for f in os.listdir(d)]


# Enter Qt application main loop
#print html_date_dir

# Display the Web file
g_pwd = os.getcwd()
g_today_string = time.strftime("%Y_%m_%d", time.localtime(time.time())); 
        
print "Enter a directory name (eg. %s):" %(g_today_string)
date_dir = raw_input(">")
date_dir = date_dir.strip()

if len(date_dir)<2:
    html_date_dir = g_pwd + '/html/'+ g_today_string +'/'
    wav_date_dir = g_pwd + '/wav/'+ g_today_string +'/'
else:
    date_dir = time.strftime("%Y_%m_", time.localtime(time.time()))+ date_dir
    html_date_dir = g_pwd + '/html/'+ date_dir +'/'
    wav_date_dir = g_pwd + '/wav/'+ date_dir +'/'

html_file_list = listdir_fullpath(html_date_dir)
# Create a Qt application
if __name__=='__main__':
    app = QApplication(sys.argv);
    brower = DisplayHtml(html_file_list);
    sys.exit(app.exec_());

\end{verbatim}
\hfill {\tiny  edited on 2012-05-27.}
%
\section{GDB, Assembly}
\begin{verbatim}
(gdb) set disassembly-flavor intel
(gdb) set disassemble-next-line on
(gdb) b *0x08048431
(gdb) ni
(gdb) disass main
(gdb) si
\end{verbatim}
\hfill {\tiny  edited on 2012-05-28.}
%
\section{Prey, Email, Phone, SMS}
Yesterday, the unhappy events happened. 
I have to monitor the following things that will happen next.
I enabled the 'Prey' to collect the screen-shot and forwarded the notice mail from G-mail to to 139 email.
If any emails arrive, the 139 Email system will notice me via SMS. That is it.
\hfill {\tiny  edited on 2012-05-29.}
%
