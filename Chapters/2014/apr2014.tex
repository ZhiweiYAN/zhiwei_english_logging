
\chapter[April - 2014]{April - 2014} % Chapter title

\label{ch:apr:2014} % For referencing the chapter elsewhere, use \autoref{ch:introduction} 

%----------------------------------------------------------------------------------------
\section{What if you didn't need money or attetion?}
\marginpar {\tiny edited on 2014-04-01.}

We do so many things for the money. 
It's so deeply built into our culture that it takes a real effort to 
realized it's the reason behind so many of our actions.

If you stop doing all these things you're just doing for the money, 
or the attention, what's left?

\section{Big data  are we making a big mistake}
\marginpar {\tiny edited on 2014-04-02.}
Big data: are we making a big mistake?
By Tim Harford

\url{http://www.ft.com/cms/s/2/21a6e7d8-b479-11e3-a09a-00144feabdc0.html#axzz2xS1VXiUc}

Sometimes, {\bf Big Data analysis} produces uncannily accurate results; They are cheap to collect relative to their size and they are a messy collage of datapoints collected for disparate purposes and
they can be updated in real time. 

\begin{enumerate}
\item cheap to collect
\item collage of datapoints
\item disparate purposes
\item update in real time
\end{enumerate}

With enough data, "the numbers speak for themselves".

As our communication leisure and commerce have moved to the internet and the internet has moved into our phones, our cars and even our glasses or watches,  life can be recorded and quantified in a way that would have been hard to imagine just a decade ago.
There are a lot of small data problems that occur in big data, says Spiegelhalter.  "They don't disappear because you've got lots of the stuff. They get worse."


We seek new ways to understand our lives.
Figuring out what causes what is hard (impossible" some say).
If you have no idea what is behind a correlation, you have no idea what might cause that correlation to break down. 

Statisticians have spent the past 200 years figuring out what traps lie in wait when we try to understand the world through data. 
They cared  about correlation rather than causation.
When it comes to data, size isn't everything.

US-based Twitter users were disproportionately young  urban or suburban and black.
There must always be a question about who and what is missing" especially with a messy pile of found data.
Twitter users are not representative of the population as a whole.
Who cares about causation or sampling bias" though" when there is money to be made?
We found data contain systematic biases and it takes careful thought to spot and correct for those biases. 

There are two issues: sample error and sample bias. The larger the sample, the smaller the margin of error (sample error). 
The sample bias is more dangerous, because the sample is not randomly 
choosen at all. 
You should find an unbias sample. Find the detail in {\it Number sense}, wroten by Kaiser Fung.
Another bias example is about the iPhone app, {\it Boston Street Bump}.
You should find data contain systematic biases and it takes careful thought to
spot and correct for those biases.

{\bf Multiple-comparisons problem: }
We observed pattern could have emerged at random, we call that pattern "statistically significant".
"Why Most Published Research Findings Are False"
There are a few cases in which analysis of very large data sets has worked miracles, like
Google Translate.

Big data do not solve the problem that has obsessed statisticians and scientists for centuries: the problem of insight , of inferring what is going on, and figuring out how we might intervene to change a system for the better.  To use big data to produce such answers will require large strides in statistical methods.
"But nobody wants 'data'. What they want are the answers."
Many contrary results are languishing in desk drawers because they just didn't seem interesting enough to publish. 

{\bf{"Big data"}} has arrived but big insights have not. The challenge now is to solve new problems and gain new answers -- without making the same old statistical mistakes on a grander scale than ever.
As for the idea that "with enough data" the numbers speak for themselves" -- that seems hopelessly naive in data sets where spurious patterns vastly outnumber genuine discoveries.

\section{New RFS -- Breakthrough Technologies}
\marginpar{\tiny edited on 2014-04-02.}

{\bf Great companies:}
\begin{enumerate}
\item be with a series of small wins that compound over time
\item focus on solving real problems for real customers, and not just developing technology for its own sake.
\item not to bite off an initial project that is far too big and expensive.
\end{enumerate}

{\bf Breathrough Technologies}
\begin{enumerate}
\item Energy: Cheap energy, energy storage and transmission.
\item AI: 
\item Robotics: We count a self-driving car as robot.
\item Biotech:
\item Healthecare: preventative healthcare.
\item Education: Using the Internet to distribute traditional content to a wider audience. One-on-one in -person interaction.
\item Internet Infrastructure: The Internet is a transformative power. We hope use the Internet to fix government. An important trend is the API-ification of everything. As more and more businesses are accessible with a web API, the Internet becomes more and more powerful.
\item Science: new business models for basic research.
\item Transportation and housing. About half of all energy is used on transportation, and people spend a huge amount of time unhappily commuting.
\end{enumerate}

\section{Bored People Quit}
\marginpar {\tiny edited on 2014-04-22.}
Bored People Quit
By  
\par \url{http://randsinrepose.com/archives/bored-people-quit/}

{\bf Situation}
These are the people who show up when your single best engineer casually
and unexpectedly announces, ''I'm quitting. I'm join my good friend to found a start-up.
This is my two weeks' notice''.

{\bf Detcting Boredom}
\begin{enumerate}
\item A decrease in productivity is a great ealy sign.
\item You ask, "Are you bored?"
\item They tell you. And you listen.
\end{enumerate}

In reality, the boredom was a seed.
You always need to be able to answer two questions regarding each person on your team:
\begin{enumerate}
\item Where are they going?
\item What are you currently doing to get them there?
\end{enumerate}

There's no shit work when the work is all yours,
there's just work you like to do and work you have to do.
Occasinal stints on the latter are a good perspective reset for everyone on the team,
but being left too long on "have to" work is a guarantee of evental boredom.

Random meetings, phone calls, interviews disturb your team members.
Your attention is only half the solution. 
The other half is regularly keeping folks in the loop regarding your thoughts.

My gig is the care and feeding of engineers, 
and their productivity is my productivity.
If they all leave, I have exactly no job.
Your job is to help your team succeed.

\section{Don't Fuck Up the Culture}
\marginpar {\tiny edited on 2014-04-22.}
Title: Don't Fuck Up the Culture \newline
Author: Brian \newline
URL: \url{https://medium.com/p/597cde9ee9d4} \newline

{\bf Culture}
I thought to myself, how many company CEOs
are focused on culture above all else or their culture?

Culture is simply a shared way of doing something with passion.

Our culture is the foundation for our company. 
The thing that will endure for 100 years, 
the way it has for most 100 year companies, 
is the culture.

The culture is what creates the foundation for all future innovation. 
If you break the culture,
you break the machine that creates your products.

When the culture is strong,
you can trust everyone to do the right thing.
People can be independent and autonomous.
The stronger the culture,
the less corporate process a company needs.
In organizations (or even in a society) where culture is weak,
you need an abundance of heavy, precise rules and processes.

You make decisions by comparing to culture with your other targets.
If compared to culture,
they are relatively short-term.
These problems will come and go.
But culture is forever.

\section{Game servers: UDP vs TCP}
Title: Game servers: UDP vs TCP, \url{http://1024monkeys.wordpress.com/2014/04/01/game-servers-udp-vs-tcp/}
\newline
Author: Christoffer Lerno \newline

The most damning property of TCP is the congestion control.
You use reliable UDP instead of TCP - to get rid of its congestion contol.

\begin{enumerate}
    \item Use HTTP/HTTPS: if you are making occasional, client-initiated stateless queries and an occasinal delay is OK.
    \item Use persistent TCP sockets: if both client and server may independently send packets but an occasional delay is OK.
    \item Use UDP: if both client and server may independently send packets and occasional lag is not OK.
\end{enumerate}

These are mixable too. Your MMO might first use HTTP to get the latest updates, then connect to the game servers using UDP.

Never be afraid of using the best tool for a task.

Starting an action before confirmation is a typical latency/lag hiding technique.


\section{How to become a greater developer?}
\marginpar {\tiny edited on 2014-04-23.}
Title: How to become a greater developer,  \url{http://peternixey.com/post/83510597580/how-to-be-a-great-software-developer}\newline
Author: Peter Nixey \newline


Your seniority and value as a programmer is measured not in what you know,
it's measured in what you put out.
The two are realted but definitely not the same.
Your value is in how you move your project forward and 
how you empower your team to do the same.

You should aim for simplicity.
Simplicity is far more easily attained by time spent working and refactoring than hours of pure thought and 'brilliance'.

Simplicity and excellence are most reliably attained by starting with something, 
anything that gets the job done and reworking back from that point.

Companies are built on people and teams who day in, day out, 
commit good code that enables others do the same.
Great product is built by work horses, not dressage hourses.

Not only is their output erratic but their superiority is aspiratinal and infectious.
Their arrogance bleeds toxically into the rest of the team.
Your projects depend on reliable people who work in reliable work.
Greate developers are not people who can produce bubble sorts or link shorteners on demand.
They are the people who when harness them up to a project,
never stop moving forward and inspire everyone aroubnd them to do the same.
You do not need {\it RockStars}.

{\bf Name your functions and variables well} \newline
It is one of the MOST IMPORTANT skills in programming.
Function naming is the manifestation of problem definition 
which is frankly the hardest part of programming.

Names are the boundary conditions on your code. 
Names are what you should be solving for.
If you name correctly and then solve for that boundary conditions 
that the names creates you will almost inevitably be left with highly function code.

Function names create contracts between functions and the code that calls them.
Good naming defines good architeture.
Good function and variable naming makes code more readable and tightens the thousands of contracts which criss-cross your codebase.
Sloppy naming means sloppy contracts, bugs, and even sloppier contracts built on top of them.

Notice how much stronger this approach is than using comments.
If you change the logic there is immediate pressure on you to change the variable names.
Not so with comments.

{\bf Go deep before you go wide - learn your chosen stack inside out} 
\par 
The marjority of the things you are trying to do have already been solved by the very stack you are already using.
Most programmers waste huge amounts of time by lazily re-creating implementations of pre-exsiting functionality.
You should undestand those existed technologies well that you use everyday.


{\bf Learn from those good codes}
\par
The grand chessmasters spend proportionally much more time studying previous other good chess player's games than the average players.
You should develop an aesthetic appreciation for code.
Simplicity is beatuiful and simplicity is what we want.

The truth is that the truth is sometimes ugly but you should always strive for beauty.

Your code has two functions: the first is its immediate job.
The second is to get out of the way of everyone who comes after you
and it should therefore always be optimised for readability and resilience.

{\bf Weight features on their lifetime cost, not their implementation cost}
\par
Features and architecture choices have maintenance costs that affect everything you ever build on top of them.
Abstractions leak and the deeper you bury badly insulated abstractions the more things will get stained or poisoned when they leak through.
Experimental architecture and shinny features should be embarked on every carefully and only for very good reasons.

Build the features you need before the features you want and be VERY careful about architecture.

{\bf Leverage of Technical Debt}
\par 
Einstein once said that " there is no force so powerful in the universe as compound interst'.  
If you do not absolutely need them; 
do not write them.
You are in an exploratory phase. 
You will pivot both on product and on technical implementation.
Your initial code is scouting code.
It should move you forward fast, illuminate the problem and the solution and give you just enough space to build camp.

Check and re-check your code.
You should make sure your code works. 
It's not the testers' job and it's not your team-mates'job.
It's your job.
Lazily written code slows you down, increases cycle times, releases bugs and pisses everyone off.

Do actual work for at least (only) four hours every day, and you will be one of the best contributing members of your team.
Proper work is work that includes no email, no hacker news, no meeting, no dicking around. It means staying focussed at least 45 minutes at a time.

{\bf Write up the things you've done and share them with the team}
\par
For example, have a tough time getting a fresh install of Postgres or ImageMagik to work? If you found it hard, the rest of your team will probably also find it hard so take a moment to throw down a few paragraphs telling them what you did and saving them the time next time.

Think of testing like armour. 
The more of it you wear the harder it is to hurt you but the harder it is to fight too.

{\bf Make your team better}
\par
Does your presense make your team better or worse? 
Does the quality of your code, your documentation and your technical skills help and improve those around you?
Do you encourage and inspire your team-mates to become better developers?

{\bf Who are you?} \par
It's not who you are underneath, it's what you do that defines you.

%%%%%%%%%%%%%%%%%%%%%%%%%%%%%%%%%%%%%%%%%%%%%%%%
