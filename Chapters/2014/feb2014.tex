
\chapter{Feb - 2014} % Chapter title

\label{ch:feb:2014} % For referencing the chapter elsewhere, use \autoref{ch:introduction} 

%----------------------------------------------------------------------------------------
\section{Handle Excel files in C}
Perhaps, we begin to handle Microsoft Excel files in C.
We do the search over the Internet.
There are several packages or libraries that can be used.
\begin{enumerate}
\item xlsLIB, C++/C library to construct Excel .xls files in code. 
Detail in \url{http://sourceforge.net/projects/xlslib/files/}
\item Office (2007) Open XML File Formats. Detail in \url{http://msdn.microsoft.com/en-us/library/aa338205.aspx}
\item CSV formats. The format do not handle Chinese characters very well, 
since it includes the information of encoding in itself.
\end{enumerate}
The last two methods are recommended strongly. 

\section{20 Years of Web Development}
I read the ariticle {\it 20 Years of Web Development}, written by Reuven M. Lerner, 
a well-known columnist of Linux Journal.
The idea from the article make me feel something inside, same as the author.

If I can not get my fix of e-mail, or blogs, or newspapers or Hacker News,
I feel cut off from the rest of the world.
I'd like to take the opportunity to look at where Web technologies have come from,
and where they're going.

\subsection{In the Beginning}
In the beginning, the Web was static.
A Web brower requested a document from the server.
Instead, the server could lie to the browser,
creating a document on the fly, by executing a program.
The Web servers invoke external programs in a 'cgi-bin' directory.
The 'cgi' programs could be written in C and shell, 
later, in Perl and Tcl in those days.

For example, MIT students put their newspapers on the Web in 1993,
and made it possible for people to search through the newspaper's archives.

In 1995, no one really thought the web programming as serious software development.
At that time, applications were serious desktop software. 
I began to understand where the Web was head when you learned about relational databases
and understand how powerful Web applications could be when connected to a powerful,
flexible system for storing and retrieving data.
DB ranking list is at \url{http://db-engines.com/en/ranking}.

And at that point, the browser could be the beginning of a new platform for applications.
[PPTs also turn the bullets scheme into words-scheme.]
Nowadays, the vision has turned into a reality.
\par\hfill {\tiny edited on 2014-02-27}

\subsection{Libraries and Frameworks}
It did not take long before Web development really began to take off.
The growth of the Web happened at about the same time that the term {\it open source} was coined.
Open-source operating systems and languages - in those days,
Perl, Python and PHP - grew in popularity, both because they were free of charge
and because they offered enormous numbers of standardized, debugged and highly useful
libraries.


%%%%%%%%%%%%%%%%%%%%%%%%%%%%%%%%%%%%%%%%%%%%%%%%
